\documentclass[../main.tex]{subfiles}
\begin{document}

%Dopo due anni di questo persorso di laurea magistrale, mi ritrovo a scrivere i ringraziamenti, che per me è come scrivere le conclusioni di questo cammino universitario, di questa vita da studentessa fatto di incroci, di stop, di semafori verdi, rossi, di salite e di discese.

%Ma partiamo dalla tesi che è più semplice. Innanzitutto, mi sento di dover ringraziare per la preziosissima collaborazione la professoressa Stephanie Novak, per i suoi consigli, per il suo tempo dedicatomi, per il suo aiuto.

%Questa tesi però racchiude due anni di esperienze, di nuove conoscenze, di successi e di fallimenti, di vittorie e di cadute.

%Innanzitutto, Venezia non sarebbe stata possibile senza il sostegno di mamma e babbo. All'inizio è stata dura vedere la "piccola"(ma neanche troppo) di casa andare via, ma mi hanno sempre dato il loro appoggio, non facendomi mancare mai niente. Anche da lontano, mi sono stati vicini, hanno creduto in me. Se sono stata capace di vivere da sola a Venezia (e anche in Kosovo se è per questo) è perché voi mi avete fornito tutti gli strumenti necessari per far sì che potessi volare da sola. Quindi grazie. Potrei scrivere una tesi intera per ringraziarvi e non sarebbe abbastanza...quindi fatevi bastare questo per ora!
%E poi ci sono loro, i miei fratelloni, che a modo loro mi dimostrano il loro affetto. Nonostante le difficoltà, le divergenze, i caratteri diversi, alla fine siamo sempre qui, a volerci bene incondizionatamente.

%Non potrei andare avanti senza ringraziare la mia Testa. Entrato con passo felpato nella mia vita, senza neanche che me ne accorgessi, mi ha arricchito con la sua visione leggera e curiosa della vita. Mi hai preso per mano e mi hai incoraggiato a essere una versione migliore di me, mi hai spronato e mi hai accompagnato, senza lasciare mai la presa. Mi hai rassicurato quando pensavo di non farcela e mi sei stato accanto anche quando eravamo lontani. Lontani ma mai distanti. Grazie per avermi fatto il tifo durante questa magistrale e durante la stesura di questa tesi, grazie di aver creduto in me quando non mi sentivo abbastanza, grazie per avermi dato quegli abbracci che ti fanno sentire a casa quando mi sentivo persa. Grazie per avermi teso la mano quando mi sentivo cedere, quando durante i momenti difficili mi davi un po' di luce nell'oscurità. E grazie anche di farmi apprezzare i momenti belli, di farmi apprezzare il presente per quello che è. Grazie di farmi compagnia in questo colorato, impegnativo, straordinario cammino della vita. Wfa wfa.

%Un'altra persona che mi accompagna da mooooooolto tempo è Ila, la mia tata. In questi ultimi due anni oltre che a amiche, confidenti, sorelle, psicologhe, e chi più ne ha più ne metta, siamo diventate anche coinquiline. Non so neanche dove cominciare a spiegare quanto sia grata di ciò che abbiamo condiviso. Le serate da vecchiette come piacciono a noi, le confidenze dette l'una della stanza dell'altra, gli abbracci improvvisi ma necessari, i consigli, la spesa (le offerte della Coop!!), la stessa ossessione per le pulizie. Tutto. Ho apprezzato tutto ciò che abbiamo vissuto insieme. Non so dove ci porterà la vita, ma custodirò con cura questi ricordi insieme. Quindi grazie, di esserci sempre, da sempre.

%Venezia mi ha fatto il regalo di farmi conoscere persone straordinarie. Prima di tutte, Marta. Ci siamo conosciute il primo giorno di lezione, per caso abbiamo mangiato insieme e da allora non ci siamo più separate. Abbiamo affrontato insieme la magistrale e non potevo chiedere una miglior compagna! Grazie per tutte le volte che mi hai adottato e per tutti i ginseng, le pause, le risate a cuor leggero, le sedute psicologiche, le sessioni di studio matto e disperato. Sono davvero grata di come un "Hey ciao, sai dove fa lezione Basosi?" abbia dato inizio ad un'amicizia spontanea ma solida.
%E poi non potrei che ringraziare Giorgia, per la sua tenacia, la sua perseveranza, la sua dolcezza. Sempre una parola giusta nei momenti giusti. Ed Elisa, con il suo carattere frizzante, la sua testadaggine, la sua visione filosofica delle cose, che ci faceva venire dubbi più profondi sulle cose più apparentemente semplici. Grazie a voi ho veramente fatto tesoro persino delle giornate studio più intense, delle sconfitte e delle fatiche. Affrontare insieme gli scogli mi ha fatto capire che il lavoro di squadra fatto con le persone giuste ti porta ovunque. E ovviamente spero di non finire mai di spettegolare con voi. Belle le mi' pettegole!

%Pensando al periodo veneziano non posso che ringraziare anche Carolina, dolce biondina dallo spirito gentile, per le lunghe chiacchierate a cuore aperto. Dori,  

\end{document}