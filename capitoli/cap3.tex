\documentclass[../main.tex]{subfiles}
\begin{document}

In the previous chapter, there was an analysis of the concept of culture. The hope is that the analysis will provide the reader some tools and definitions to better comprehend this chapter. Indeed, in these next pages, there will be an analysis of the literature concerning the influence of culture on negotiation. The main subject of this work is international \textit{diplomatic} negotiations but, since culture affects negotiations on common features regardless of whether it is a diplomatic one or not, also scientific works on other types of negotiations will be included as a theoretical base for this thesis.%riguarda discorso

Considering the remarkably vast literature that exists on the subject, I decided to categorise the literature contemplated %se riesci a trovare un sinonimo è meglio
in this thesis following the four streams of thought identified by Faure, in his article "The Cultural Dimension of Negotiation: the Chinese Case" (1999). In the article, he addresses the work of Hofstede and also tries to categorise the cultural studies. Particularly, Faure identifies four main streams of approaches on the research on international negotiation that focuses on cultural aspects.
\begin{enumerate}
\item \textbf{Structural-processual approach}: it relies on the model of Sawyer and Guetzkow. The result is a combination of a set of factors that are contextual or situational, processual or behavioural, strategic or related to the outcome. Under this stream, culture is perceived as either integrated among contextual factors or supposed to operate directly within the categories.
\item \textbf{Behavioural approach}: under this approach, two different methodological traditions have been established. The first one aims at assessing the impact of culture on a number of behavioural variables. The second one is based on surveys and focuses on describing the influence of culture on the behaviour of the negotiators.
\item \textbf{Cognitive-strategic approach}: it intends to study the connection between a negotiator's action and their mindset in order to disclose the logic implemented during the negotiation.
\item \textbf{Stages approach}: within this framework, negotiations are divided in stages with different requirements. The satisfaction of the requirements in each stage allows the the adjustment on the different sequences and the reaching of the agreement.
\end{enumerate} %riguarda tutto perché va citato ammodo eh!!!

Faure already includes some authors in his article, dividing them in the categories he has identified. However, for the authors that are not mentioned in the article and that I have arbitrarily considered and included in this thesis, there will be an explanation for the reason why they are inserted within a specific approach.

\end{document}