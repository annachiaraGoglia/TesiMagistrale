\documentclass[../main.tex]{subfiles}
\begin{document}
\label{chap:2}

In this chapter, there will be a brief introduction to what an international negotiation is in order to welcome the reader in the context of this thesis. Only the salient aspects will be covered, as this chapter serves as a base to understand better the issue of the whole work and also to help the reader to build some tool in order to make their own opinion about the subject of this thesis.

In order to define what an international negotiation is, it is convenient to define the concept of "negotiation". To negotiate means, according to the Merriam-Webster dictionary, "to confer with another so as to arrive at the settlement of some matter"\footnote{\textit{Definition of Negotiate}. Merriam-Webster \url{https://www.merriam-webster.com/dictionary/negotiate}}. Hence, a negotiation is a process, a sequence of actions, in which the actors involved will try to reach a situation that is superior to the \textit{status quo} situation. More specifically, a negotiation is an exchange of credible threats and promises whose aim is to find an agreement. An international negotiation is a negotiation that involves international actors, such as States, Non Governmental Organizations, Supranational Institutions (such as the European Union), etc. International negotiations are growing in number because States are becoming more and more interdependent and because there are more interdependent issues States have to negotiate about: for example, human rights is a field that requires many negotiations and discussions.

\textbf{Actors.} The actors in a negotiation process can be either internal or external. The internal actors are the \textit{parties}, which are directly involved in the negotiation process and in its outcome: they have interests involved in the negotiation and they stand on a position during the process. Ideally, the position should be representing someone's interest, but a clarification needs to be made:
\begin{itemize}
\item Position: is what the actors say they want, that is what the actor choose to share with the other party. Positions are surface statements of where a person or organisation stands, and rarely provide insight into underlying motivations, values or incentives\autocite[]{watershed}.
\item Interest: is what really moves the process behind the fuss of position; the interest is a party's underline reason for the initiation of the negotiation in the first place and it reflects the values and the motivation of the parties. An interest can explain why a party choose a certain position but it does not mean that the position and the interest beneath it overlap.
\end{itemize}. The external actors, also known as \textit{third parties}, form a part of the negotiation structure but they are not directly involved in the negotiation: they don't have direct interest but they contribute to shape the outcome of the negotiation. There are three types of third parties: the \textit{Chair} of a negotiation, the \textit{Mediator}, and the \textit{External parties}.

The Chair makes sure that the actors respect the rules of the procedures; also, it gathers the information on the positions of the parties and makes sure that the negotiation process is fair. Since this position is very powerful, sometimes the Chair can exploit the position to manipulate the outcome of the negotiation.

One famous example is the behaviour of Valéry Giscard D'Estaing during the negotiation process for the drafting of a European constitution. In order to achieve a solution, a convention was established, specifically the Convention on the Future of Europe, whose Chairman was Giscard D'Estaing. He was not neutral, as he manipulated the agenda by using many tactics (the definition of \textit{tactic} will be provided later in the chapter): he used a combination of time constraint, gate keeping, and restrictive rules of procedures. He would set the agenda by keeping the proposals that he wanted to be adopted for as the last discussion topics so that they would not have the time to discuss them properly (\textit{time constraint}). He would meet with his fellow committee members and they would scan all the amendments proposed by the member states: Giscard D'Estaing then would write a summary of all the amendments; however, member states had no guarantee that the aforementioned summary would include every amendment they gave him. Indeed, the former French President took advantage of the situation so the member states would receive the draft that included the amendments that D'Estaing liked more so that the outcome of the convention would be closer to his preferences (\textit{gate keeping}). In addition, he took the arbitrary decision to allow amendments only to some member states (\textit{restrictive rule of procedures}). This demonstrates how powerful the chair can be, especially because it is not accountable to anybody.

The Mediator has many functions and usually has no official mandate. Ideally, the position of mediator should fulfill two conditions: the first one is \textit{neutrality}, that means that the mediator should not be biased as a result of having links with one of the parties; the second one is \textit{impartiality}, that is the mediator should not show any preference for the positions takes by the parties \autocite[6]{ropers1997roles}. The role gives the mediator the possibility to ease the negotiation process in many ways: the mediator can facilitate the communication between the parties by understanding each point of view and explaining it to the opposite party, can suggest initiatives, can propose modification of the game rules; furthermore, because they can avoid direct confrontation between the parties, they can act as "face-savers" for the losing party.

The external parties do not have an official role in the negotiation process. %cerca roba su external parties!!!

As the negotiation is a complex process, it can be divided in many ways. One way to look at a negotiation is to divide it in four main stages:
\begin{enumerate}
\item \textbf{Status quo}: is the initial situation. It is usually a situation less favourable than the one after the negotiation is concluded. Nonetheless, sometimes actors prefer the status quo rather than an outcome that could be less favourable for them.
\item \textbf{Verbal exchange}: is the reason why and in what ways the parties talk to each other. In this stage, information are exchanged between the parties.
\item \textbf{Strategy}: is the intellectual line of action that a party will adopt in order to maximise its payoff of the negotiation. According to Pruitt, there are four types of negotiations strategy, as he explains in his article "Strategic Choice in Negotiation"\autocite[167]{pruitt}.
\begin{itemize}
\item \textbf{Yelding}: a party will diminish its own demands and will concede. Therefore, an agreement will be found, but at the expenses of the party that implements this strategy. This strategy is implemented mostly when the the issue is not relevantly important and there is the presence of high time pressure.
\item \textbf{Problem Solving}: it is a strategy based on collaboration. By adopting this approach, parties will cooperate in order to identify solutions that will satisfy everyone's goals. In order to do so, various formulas are available, such as: expanding the pie (parties will find a way to increase resources that have been in short supplies), cost cutting (one party gets what it wants by cutting the other's cost of conceding), compensation (a party rewards the other for conceding), logrolling (each party concedes on issues that have low priority to themselves), and bridging (a new option is developed that satisfies both parties' aims) \autocite[168]{pruitt}.
\item \textbf{Contending}: a party will pursue its goal without making concessions by tying to make the other party to concede. By adopting this type of strategy, the actors show themselves as competitive: this results in a lower chance to find an agreement than it would have been by adopting other strategies.
\item \textbf{Inaction}: it means that the parties will do as little as possible. This type of strategy often wastes time and sometimes it even suspends temporarily the negotiation\autocite[172]{pruitt}.
\end{itemize}
\item \textbf{Tactic}: all the actions used to implement the strategy adopted. The entirety of all the tactics used by the parties compose the strategy.
\end{enumerate}

Another way to look at a negotiation process is provided by Richard Shell in his book \textit{Bargaining for Advantage}, where he describes a negotiation as "an interactive communication process that may take place whenever we want something from someone else or another person wants something from us"\autocite[24]{shell}; he divides the process in four stages as well, but differently. Indeed, according to him, the stages are preparation, exchanging information, bargaining, and closing and commitment \autocite[134]{shell}.
%citalo anche per cultural issue (pg 20 del libro suo)

Preparation is the stage during which the parties do researches about the issue, the standards and principles through which they can actually find a common ground, and the perceived needs of the other party involved in the negotiation process.

Exchanging information is the most important step, according to Shell. Indeed, this step aims at providing the parties some tools to understand each others' point of view by discerning each others' underlying interests and objectives.

The bargaining stage is the stage that is mostly associated to the actual negotiation process by most people, but this stage and negotiation are not equivalent. Though this stage is the one where the most part of the work is done by the parties \autocite[324]{culo}.

The closing and commitment part ends the actual process. Once the parties have found an agreement and they conclude that other alternatives they explored had not more value than the outcome they found. The parties trust in each other for each to fulfil their promise. 

Moreover, another contribution on the division of the negotiation process is given by Zartnam and Berman in their book \textit{The Practical Negotiator}. They describe the negotiation process as a process in which divergent values are combined into an agreed decision, and it is based on the idea that there are appropriate stages, sequences, behaviours, and tactics that can be identified and used to improve the conduct of negotiations and better the chances of success \autocite[2]{zartman1982practical}. They divide a negotiation process in three stages: diagnostic phase, formula phase, and detailed phase. These stages are not isolated, indeed they sometimes overlap.

The diagnostic phase feature the definition of the situation in which the parties explore the possibilities of negotiating. According to Zartnam and Berman, a negotiation will be due when the a situation that is already not positive for the parties will become worse in the future if no action is taken. This feeling must be shared by both parties in order for the negotiation to start. Furthermore, the parties will have to agree that only by making a joint effort they can reach an outcome that is favourable for both parties' interests. Zartnam and Berman assert that, in order to identify problems and behaviours of each stage, appropriate behaviours "during the diagnostic phase involve: to be knowledgeable of facts of the problem, to have thorough information concerning similar issues, particularly their precedents and referents, to be clear about the context and perceptions important for both parties, and to always be able to think alternative solutions" \autocite[98]{garcia}.

In the formula phase, the parties have agreed to proceed with the negotiation reaching a point that is called \textit{Turning Point of Seriousness} \autocite[3]{zartman1982practical}: that is when each parties realises that the other is serious about jointly work to find a common solution. Indeed, the parties will try to search a general principle that will serve as a guideline for the entirety of the negotiation.

During the detail phase the negotiation are going to focus on the assessment of the details for the implementation of the formula they agreed. According to Zartman and Berman, one of the most important thing in a negotiation should be the high level creativity of the parties that corresponds to  the way negotiators handle concession-making situations.

The factors that influence a party in taking the decision are many: first of all, there is culture (which is the focus of this thesis), then the structural context, the situational factors, and also the personality of an actor.
\pagebreak 
\end{document}