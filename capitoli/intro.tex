\documentclass[../main.tex]{subfiles}
\begin{document}
%intracultural negotiations and transcultural negotiations 

Culture is 

Culture is one of the factors which helps us create our own identity and that guides us in our behaviours. But to which extent do culture affect our choices? Is it possible to overcome this influence?

These are the central questions of this thesis. The aim is to analyze the role of the influence of culture in diplomatic international negotiations. In particular, the focus is on how the parties of a negotiation are influenced by the culture they are embedded in. Indeed, the influence of culture on the parties behaviour deeply affects the outcome that has to be achieved.

I chose this topic because I have always been interested in the role of culture in an individual's life: it is fascinating to see how it shapes the mindset and also how it is always changing and evolving, consequently changing the mindset of people. This shaping can happen also when there are national interests at stake, such as during a diplomatic negotiation.  %???? 

The work is divided as following. First, there will be a definition of the concept of culture, how it is defined and what definition is useful for the purpose of this thesis, and a definition of what an international negotiation is, how it is structured and how the process evolves. The focus will be on the actors of a negotiation and on their relationship with culture. Then, there will be an analysis of how culture influences international negotiations with the help of practical examples: particularly, there will be an explanation of a case in which culture helped to find an agreement, and an example in which culture stood in the way and brought to a stalemate situation. Finally, there will be an assessment on to which extent culture influences the outcome and whether this influence can be overcome. %cambia questo paragrafo visto che hai cambaito la struttura

%nell'ntro si scrive anche un po' gli autori e chi andremo a citare

\end{document}