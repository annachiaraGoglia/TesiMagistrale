\documentclass[../main.tex]{subfiles}
\begin{document}
%intracultural negotiations and transcultural negotiations 
  \begin{tabular}{c}
 \textit{Preservation of one's own culture does not require}\\
 \textit{contempt or disrespect for other cultures.} (Cesar Chavez)\\
  \end{tabular}\\

%qual è la linea che delimita l'influenza della cultura e una tattica di negoziazione?

In the field of international relations, it is very important to acknowledge and respect the diversity of cultures. Especially when different cultures meet, such as in a negotiation process.

A basic assumption needs to be made before starting with this thesis. We are assuming that the context within which the actors of the negotiation are framed allows for individuals to be influenced by culture.
Indeed, some international relations theories consider the individuals as rational (the game-theory approach, for example), therefore they could not be influenced by anything other than reasoning. The approach considered for this thesis is the neo-liberalist one.

Culture is one of the factors which helps us create our own identity and that guides us in our behaviours. But to which extent do culture affect our choices? Is it possible to overcome this influence?

These are the central questions of this thesis. The aim is to analyse the role of the influence of culture in diplomatic international negotiations. In particular, the focus is on how the parties of a negotiation are influenced by the culture they are embedded in. Indeed, the influence of culture on the parties' behaviour affects the outcome that has to be achieved.

But what does "influence" mean in this context? Influence means that the role of culture can be traced back


I chose this topic because I have always been interested in the role of culture in an individual's life: it is fascinating to see how it shapes the mindsets of the individuals and also how it is always changing and evolving. This shaping can happen also when there are national interests at stake, such as during a diplomatic negotiation.

At first, the aim was to analyse the degree with which culture influences the negotiations and, then, to provide some practical example. However, the scientific resource turned out to be so extensive, that the structure of the thesis was resettled to be an analysis on the influence of culture across research.

Indeed, when it comes to culture, there is so many things to acknowledge that it is necessary to discern which aspects to consider and to analyse. For example, culture can mark the time to have a meal: in some cultures, lunch time is at noon, in some other is at one pm \autocite[7]{helen}; or it can tells us what to wear in formal occasions.

Of course this work does not have the presumption to be an exhaustive collection of every article or book that has been written about the influence of culture in international negotiations. Indeed, it aims at making some order on the subject by following the streams identified by Faure.

The difficult thing about discussing culture is to be careful not to transcend into stereotypes. Indeed, when dealing with cultural elements it is easy to generalise and let stereotypes prevail. However, in the works considered in this thesis, every cultural trait was explained not through stereotypes but through a reasonable explanation. Some of the results were matching with known stereotypes, however they were not dealt with as such. 

This work does not have the presumption to be exhaustive; however, the hope is that it will make some order on the subject.\\

This thesis is divided as follows. First, there will be a chapter which will guide the reader through the chapters by giving them a background contextualisation and a  conceptual map to follow the road of the work. 
This chapter is divided in three main parts.

First, since we are dealing with international negotiations, there will be an introductory presentation of what is a negotiation, what are the actors, what are the stages of a negotiation, what is a strategy, what is the dynamic behind the position of a negotiator. Of course, there are many aspects of a negotiation to consider: what it means to have power in a negotiation, what are the problems with which an international negotiator has to face both at the international level and the domestic level\footnote{For further references, see Putnam, Robert D. "Diplomacy and domestic politics: the logic of two-level games." \textit{International organization} 42.3 (1988): 427-460.}, what is behind a back channel negotiation\footnote{For futher references, see: Wanis‐St. John, Anthony. "Back‐channel negotiation: International bargaining in the shadows." \textit{Negotiation Journal} 22.2 (2006): 119-144.}.
For the purpose of this thesis, the definitions and notions given in the chapter are enough to contextualise the work.
The authors called into question on this matter are Pruitt for his contribution with his article "Strategic Choice in Negotiation" (1983) and Zartman and Berman and their book "The Practical Negotiator" (1982).

In the second part of the chapter, there will be an analysis of the cultural aspect of this work. It is not possible to give an exhaustive definition of culture, in my opinion. However, its facets can be considered and used within the framework of the thesis. The article from which I took inspiration to give an orderly scheme to the chapter is "What is Culture? A compilation of Quotations"(2012) written by Helen Spencer-Oatey. Through this article, it was possible to give some insights of the concept of culture. I did not take every point she made as it was too far from the purpose of this thesis. 
First of all, as Schein suggests culture can manifest itself in many layers \autocite[3]{schein}. Indeed, it is possible to assess that culture can be perceived through visible artefacts, such as behaviours, through values, or through basic assumption that were once values and that slowly became taken for granted and pre-concious. Hofstede adds that there are three levels of uniqueness in the human mental programming \autocite[6]{hofstede}: human nature, culture, and personality. Human nature is what every human has in common. Sometimes culture can be confused with personality, however they are not the same. Culture stands between the human nature and the personality.
Also, Hofstede identifies many level of cultures, such as national level; regional/ ethnic/ religious/ linguistic affiliation; gender level; generation level; role category; social class level; organisational or corporate level \autocite[18]{hofstede}.
However, to address the issue of this thesis, it is sufficient to distinguish just one level: the national one. This is because, in international negotiations, negotiators are thought to belong to the same (more or less) organisational, generational, social class category.

Third, there is an explication of how the thesis is structured. Indeed, as the scientific literature found on the subject was vast, I decided to follow Faure's structure of his article "The Cultural Dimension of Negotiation: the Chinese
Case" (1999). I chose it because Faure identifies four approaches with which various authors have dealt with the influence of culture in negotiations. Other than the authors cited by Faure, I added the authors I read by putting them under the approaches I found most suitable. For every author I added, there is an explanation of the reason why I added that author under that specific approach. Specifically, those approaches are: structural-processual, behavioural, cognitive-strategic, stages \autocite[192]{faure}.

The following chapters are the unfolding of Faure's identified approaches.

Chapter 3 analyses the structural-processual approach. This approach takes into consideration the context of a negotiation. It relies on the model of Sawyer and Guetzkow (1965). Since it was not possible to trace the original book (there are only two hard copies, one in Sidney and one in Malibu), it is sufficient to say that they identify five groups of variables that intervene within the negotiation process.
The first author, Rosalie Tung, helps us to understand how culture influences negotiations in its context. It means that even before meeting with the negotiators, it is possible to assess which elements are relevant for an analysis of cultural influence.
Weiss, on the other hand, presents an analysis based on the relationships, on the behaviours, and on the conditions of a negotiation. He analyses the cultural influence from this perspective. He does this by analysing the different levels of relationships which are intertwined with the behaviours of the parties under the condition that influence the negotiations. Among those conditions, there is culture, which is the focus of the thesis.

Chapter 4 focuses on the behavioural approach. According to this approach, the influence of culture relies in the behaviours of the negotiators.
Particularly, Faure identifies two streams.
In the first stream, the focus is on the assessment of the actual influence of culture by analysing behavioural-related variables. The authors cited under this approach are Carnevale, who provides us with an analysis under the individualist/collectivist dichotomy; Kirkbride, Tang, and Westwood, who approach the issue by highlighting the Chinese values within a negotiation and assessing to which extent culture influences Chinese negotiators' behaviour. Moreover, I arbitrarily inserted Trompenaars, who gives his contribution to the topic by identifying seven cultural dimensions within a negotiation; and Hall, who uses the dichotomy of low-context and high-context cultural dimensions.
The second stream first analyses the impact of culture on negotiators' behaviours and then assesses the consequences of this impact. The authors cited are Zueva, Rogers, Corbett, and Cathro, who conducted a survey on British and New Zealanders negotiators by relying also on Hofstede's cultural dimensions.

Chapter 5 is directed at connecting the mindset and cognition of negotiators with their behaviours in order to explain what is the logic behind that.
The first authors Faure considers are Weiss and Stripp, who compare many national culture profiles (American, French, Chinese, Japanese, Mexicans, Nigerian, and Saudis) within twelve different categories; and Faure himself, who provides us with an explanation on how Chinese negotiators perceive a negotiation. I added Salacuse, who identifies ten factors through which cultures behave differently within a negotiation process.

Chapter 6 focuses on explaining the cultural dimension in the negotiation process
by dividing the negotiation in stages. 
The author Faure cites is Salacuse: he identifies three main stages (prenegotiation, conceptualisation, and detail arrangement) in order to simplify  the process.
I inserted Adair and Brett, who identify four stages of a negotiation (relational positioning, identifying the problem, generating solutions, and reaching agreement) and assess how culture influences each stage.


%,continua con spiegazhione dei capitoli


 %cambia questo paragrafo visto che hai cambaito la struttura

%nell'ntro si scrive anche un po' gli autori e chi andremo a citare

\end{document}