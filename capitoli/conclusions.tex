\documentclass[../main.tex]{subfiles}
\begin{document}

The central question of this thesis is "How culture influences the negotiation process?". It is a tough question because the definition of culture is difficult to outline. Culture is a fascinating yet mysterious concept that still today is object of debate. \\

In order to answer the question, I borrowed the structure of Faure's article "The Cultural Dimension of Negotiation: The Chinese Case" (1999). He identifies four streams within which research has focused on the cultural aspect of the negotiation process.

The first one takes into consideration the context in which a negotiation takes place. Tung and Weiss prove us that culture can be found in the context of a negotiation, even before it starts. Tung suggests that culture can have an influence in the decision making style of negotiators: the different concept of time and the high-context Chinese culture lead to a difference in how negotiators make their decision and how they communicate them.
Through her survey, Tung was able to assess that cultural differences are a factor responsible for the failure of a negotiation, since communication and business practices are conceived differently. Since the answer were given by negotiators, we can say that negotiators are self aware of the fact that cultural differences have a weight in the negotiation process.
Weiss delivers a study on the RBC perspective, that is relationship, behaviours, and conditions. It is indeed in \textit{conditions} that we find the cultural feature of the work: according to Weiss culture refers both to what people acknowledge in interpreting the world to behave accordingly and to all the learned behaviours. 
Weiss tells us that cultural influence can be perceived from the moment parties "warm up the engines" before starting to negotiate (for example by looking at the way the top management is involved in the negotiation) to the aftermath of the negotiation (for example by checking the compensation of the negotiators).
Every aspect embedded in the context can be accountable for the cultural influence.

The second stream takes into consideration solely the behaviour of the negotiators. Faure identifies two main branches: the first one aims at testing the impact of culture on a series of behavioural variables in order to assess the actual cultural influence; the second one in based on surveys describing the impact of culture on negotiators behaviours and subsequently analysing its consequences.
What emerges under the first branch is that culture can be conceived as a mediator between the negotiators and their behaviours \mancite\autocite[321]{carnevale}, because it stands between them and contributes actively to a negotiation. Trompenaars identifies three categories in order to assess the cultural influence: relationship with people, attitudes to time, and attitudes to the environment. Each of these categories are faced by the negotiators in a different way depending on their cultures, and some of them may have a bigger impact than others. Hall, on the other hand, provides us with an analysis of the behaviours of the negotiators through the context they belong to: he differentiates between low-context and high-context cultures. The result is that even the actual hard copy of the agreement will have a different length depending on the context of the cultures.
The results coming from the second branch are provided by a survey made on British and New Zealander negotiators from which the authors identify nine characteristic of negotiating style that might be influenced by culture. Even though United Kingdom and New Zealand are both considered Western, low-context cultures, they still had different answers on some categories. What emerges is that culture can have an influence on negotiation even on aspects that do not pop up to mind first when thinking about negotiations (such as, for example, the orientation towards a short-term goal or a long-term goal).

The third approach aims at connecting negotiators' cognition and actions in order to find a logic behind.
In order to deliver their contribution, Weiss and Stripp, through a survey made on different nationalities, identify five categories under which culture can influence a negotiation: general model, role of the individual, interaction in disposition, interaction in the process, and outcome. They extract factors which mirror cultural differences and provide us with a logic behind the answers given by the respondents. To do so, they break down each factor by explaining the reason behind a negotiator's action. What emerges is that, for example, even the predominance of a type of issue oven an other comes from the cultural framework.
Faure also contributes by focusing on the Chinese culture. He reckons that Chinese negotiation conception combines two different types of activities: mobile warfare and joint quest \autocite[140]{faure1}. The first activity is the visible part of a negotiation approach, where in Chinese culture it means to annihilate the other party by putting their counterpart in a position of discomfort. Whereas the joint quest is the underlying part that can be observed from reading between the lines: it refers to the attitude of Chinese negotiators to focus on the structure of the problem rather that on the solution itself. Those conception come from Taoism, a mindset that prevails in the Chinese culture. Behind the cultural framework lies the logic of negotiators' actions.
Salacuse identifies ten ways through which culture can affect the negotiating style: negotiating goals, attitudes to the negotiation process, personal styles, styles of communication, time sensitivity, emotionalism, agreement form, agreement building process, negotiating team organisation, risk taking \autocite[223-224]{salacuse}. Each of these ways represent a cultural feature that explains the reason why negotiators deliver specific behaviours during the negotiation process. For example, communication style is influenced by culture (low-context/high-context, indirect/direct style) and negotiators reckon the way they communicate is the best within their cultural context. Therefore, they will act through this logic. And this logic came from culture in the first place.

The fourth and final approach takes the analysis by stages. Once more, Salacuse contributes to the research by dividing the negotiation process into three stages: prenegotiation, conceptualisation, and detail arrangement. It is possible to see how different cultural factors influence each stage and, as a result, each stage will have different length depending on the culture.
Adair and Brett build a four stage negotiation model in their article: relational positioning, identifying the problem, generating solutions, and reaching agreements. They appeal to Hall's theory of high-context/low-context cultures. Therefore, under this condition, the stages more culturally influenced are the first two, as communication is key. And where communication is key, it goes without saying that culture have an influence on how the negotiation will be conducted.

\


This thesis tried to answer the question of "How culture influences the negotiation process"? Of course it does not have the presumption to %continua qui

\end{document}